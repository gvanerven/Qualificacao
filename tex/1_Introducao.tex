Ao iniciar o estudo de um tema, costuma-se realizar pesquisas na internet, ler livros, estudar exemplos, fazer exercícios e atividades práticas, além de experimentos para se ter insumos que permitam compreender melhor o que se está estudando até se alcançar um nível de compreensão sobre o tema que acredita-se ser satisfatório, como ter as condições de realizar uma tarefa.

Entretanto, para aprender algo não é comum consultar todos os livros ou realizar todos os exercícios disponível. Costuma-se selecionar um conjunto e, em alguns casos, aumenta-se de forma gradual a quantidade e complexidade, tentando direcionar para onde foram detectadas deficiências no aprendizado. Isso é o comum que os seres humanos fazem, já as máquinas costumam "aprender" de outra forma.

Em geral, os algoritmos de aprendizagem de máquina "aprendem" tentando ajustar uma função de previsão, diminuindo os erros baseados em uma massa de dados. Na maioria dos casos, busca-se ter a maior quantidade de dados possível e com uma distribuição estatística similar à população. Enquanto os humanos tentam identificar os exemplos que melhor vão levar a compreensão, a aprendizagem de máquina busca o volume tentando cobri as várias possibilidades.

Apesar da comparação de formas de aprendizagem não serem exatamente justas, já que na tarefa do ser humano é realizado toda uma compreensão e a aprendizagem de máquina consiste em processamento matemático para ajustar uma função, é sobre essa ideia, de que se pode selecionar o que aprender buscando ser mais eficiente, que surge a linha de pesquisa \textit{Active Learning}, subárea de aprendizagem de máquina.

\textit{Active Learning} parte da hipótese que é possível que o modelo escolha a melhor forma de aprender a partir de estratégias de seleção da amostra. Dessa forma, torna possível obter resultados tão bons quanto se fossem utilizados todos os dados. Essa abordagem pode então diminuir diversos custos de projetos de aprendizagem de máquina, como tempo, processamento e de obtenção de rótulos, para o caso de classificadores, por exemplo.

Esse trabalho busca explorar essa abordagem aplicando-a em uma área da mineração de dados, especificamente a classificação de textos, e espera tanto responder questões em aberto, como contribuir para melhor entendimento e expansão da área de \textit{Active Learning}. No presente capítulo, serão apresentadas as motivações para o estudo da área, assim como os objetivos e quais contribuições espera-se atingir.

%\url{http://www.escritacientifica.com/}

%%%%%%%%%%%%%%%%%%%%%%%%%%%%%%%%%%%%%%%%%%%%%%%%%%%%%%%%%%%%%%%%%%%%%%%%%%%%%%%%
%%%%%%%%%%%%%%%%%%%%%%%%%%%%%%%%%%%%%%%%%%%%%%%%%%%%%%%%%%%%%%%%%%%%%%%%%%%%%%%%
%%%%%%%%%%%%%%%%%%%%%%%%%%%%%%%%%%%%%%%%%%%%%%%%%%%%%%%%%%%%%%%%%%%%%%%%%%%%%%%%
\section{Motivação}%
Todos os cursos do \acrfull{CIC} da \acrfull{UnB} exigem a produção de um
texto científico como requisito para formação. %(veja \refAnexo{NormasGerais} para mais detalhes)
As etapas desta monografia/dissertação/tese devem seguir o \emph{método científico}.

%%%%%%%%%%%%%%%%%%%%%%%%%%%%%%%%%%%%%%%%%%%%%%%%%%%%%%%%%%%%%%%%%%%%%%%%%%%%%%%%
%%%%%%%%%%%%%%%%%%%%%%%%%%%%%%%%%%%%%%%%%%%%%%%%%%%%%%%%%%%%%%%%%%%%%%%%%%%%%%%%
%%%%%%%%%%%%%%%%%%%%%%%%%%%%%%%%%%%%%%%%%%%%%%%%%%%%%%%%%%%%%%%%%%%%%%%%%%%%%%%%
\section{Objetivos}%
Ciência (do Latim \emph{scientia}, traduzido como ``conhecimento'') é uma forma
sistemática de produzir conhecimento (via método científico), ou o nome dado a
estrutura organizada do conhecimento obtido.

O método científico é um conjunto de regras básicas de como proceder para produzir
conhecimento, criando algo novo ou corrigindo/incrementando conhecimentos
pré-existentes. Consiste em juntar evidências empíricas verificáveis baseadas na
observação sistemática e controlada, geralmente resultantes de experiências ou
pesquisa de campo, e analisá-las logicamente.

Esta ideia foi formalizada por Newton em sua obra \emph{Philosophiae Naturalis
Principia Mathematica}~\cite{newton1833philosophiae} da seguinte forma:
\begin{enumerate}
	\item Não se deve admitir causas das coisas naturais além daquelas
	que sejam verdadeiras e sejam suficientes para explicar seus fenômenos.
	\item Efeitos naturais do mesmo gênero devem ser atribuídos as mesmas causas.
	\item Características de corpos são consideradas universais.
	\item Proposições deduzidas da observação de fenômenos são
	consideradas corretas até que outro fenômeno mostre o contrário.
\end{enumerate}%

Uma abordagem para esta metodologia é seguir os seguintes passos:
\begin{description}
	\item[Caracterização do Problema:] Qual a pergunta a ser respondida? Quais
informações/recursos necessários na investigação?
	\item[Formulação da Hípotese:] Quais explicações possíveis para o que foi observado?
	\item[Previsão:] Dadas explicações [corretas] para as observações, quais os
	resultados previstos?
	\item[Experimentos:] \ \\\vspace{-2em}
		\begin{enumerate}
			\item Execute testes [reproduzíveis] da hipótese, coletando dados.
			\item Analise os dados.
			\item Interprete os dados e tire conclusões:
				\begin{itemize}
				\item que comprovam a hipótese;
				\item que invalidam a hipótese \emph{ou levam a uma nova hipótese}.
				\end{itemize}
		\end{enumerate}
	\item[Documentação:] Registre e divulgue os resultados.
	\item[Revisão de Resultados:] Validação dos resultados por outras pessoas
	[capacitadas].
\end{description}%

Geralmente se começa com a revisão sistemática, uma metodologia de pesquisa
específica para juntar e avaliar material relevante a determinado tópico~\cite{Biolchini_2005_Systematicreviewin}.

\subsection{Veja Também}
\begin{itemize}
	\item Google Acadêmico
		\\\url{http://scholar.google.com.br/}%
	\item ACM Digital Library
		\\\url{http://dl.acm.org/}%
	\item Portal \acrshort{CAPES}
		\\\url{http://www.periodicos.capes.gov.br/}%
	\item IEEE Xplore
		\\\url{http://ieeexplore.ieee.org/Xplore/home.jsp}%
	\item ScienceDirect
		\\\url{http://www.sciencedirect.com/}%
	\item Springer Link
		\\\url{http://link.springer.com/}%
\end{itemize}

Para buscar referências, \emph{The DBLP Computer Science Bibliography}\footnote{\url{http://dblp.uni-trier.de/}}
é um ótimo recurso. Veja o \refApendice{Apendice_Fichamento} para instruções
sobre como organizar as informações de artigos científicos.


%%%%%%%%%%%%%%%%%%%%%%%%%%%%%%%%%%%%%%%%%%%%%%%%%%%%%%%%%%%%%%%%%%%%%%%%%%%%%%%%
%%%%%%%%%%%%%%%%%%%%%%%%%%%%%%%%%%%%%%%%%%%%%%%%%%%%%%%%%%%%%%%%%%%%%%%%%%%%%%%%
%%%%%%%%%%%%%%%%%%%%%%%%%%%%%%%%%%%%%%%%%%%%%%%%%%%%%%%%%%%%%%%%%%%%%%%%%%%%%%%%
\section{Contribuições do Trabalho}%

\TeX\ é ``\emph{a typesetting system intended for the creation of beautiful books
 - and especially for books that contain a lot of mathematics}''~\cite{Knuth_1986_texbook},
 um sistema de tipografia muito utilizado na produção de textos técnicos devido
 a qualidade final, principalmente das fórmulas e símbolos matemáticos gerados.

\LaTeX\ é um conjunto de macros para facilitar o uso de \TeX~\cite{lamport_latex:_1994},
cujos pacotes (a maioria centralizada na rede \acrshort{CTAN}~\cite{greenwade93}), oferecem
inúmeras possibilidades. Este sistema tipográfico visa explorar as potencialidades
da impressão digital, sem que o resultado seja alterado em função de diferenças
entre plataformas/sistemas.

Em uma publicação, um \emph{autor} entrega o texto a uma editor que define a
formatação do documento (tamanho da fonte, largura de colunas, espaçamento, etc.)
e passa as instruções (e o manuscrito) ao tipógrafo, que as executa. Neste processo,
\LaTeX\ assume os papéis de editor e tipógrafo, mas por ser ``apenas'' um programa
de computador, o autor deve prover algumas informações adicionais ~\cite{Oetiker_1995_notsoshort},
geralmente por meio de marcações (comandos).

Esta abordagem de linguagem de marcação (em que se indica como o texto deve ser
formatado) é diferente da abordagem OQVVEOQVO (``o que você vê é o que você
obtém\footnote{Do inglês WYSIWYG - ``What You See Is What You Get''.}'') de programas
para edição de texto tradicionais (como MS Word, LibreOffice Write, etc.).
Apesar destes programas serem extremamente úteis para gerar textos simples, que
são a grande maioria dos documentos, eles geralmente não têm a capacidade de lidar
corretamente com documentos complexos (como dissertações ou teses), conforme ilustrado
na \refFig{latexvsword}.%

\figuraBib{miktex}{\LaTeX\ vs MS Word}{pinteric_latex_2004}{latexvsword}{width=.45\textwidth}%

Existem diversas discussões quanto ao uso de editores de texto\footnote{Por exemplo:
\emph{Word Processors: Stupid and Inefficient} \url{http://ricardo.ecn.wfu.edu/~cottrell/wp.html}},
não há um consenso quanto a melhor forma de se gerar um documento de qualidade,
e a maioria das mídias científicas disponibiliza modelos para ambas.

Mas pode-se dizer que \LaTeX\ é mais indicado para:
\begin{itemize}
	\item notação matemática;
	\item referências cruzadas;
	\item separação clara entre conteúdo e formatação.
\end{itemize}

Enquanto os editores tradicionais são indicados para:
\begin{itemize}
	\item edição colaborativa (são mais populares);
	\item produção imediata (leve curva de aprendizado).
\end{itemize}

\subsection{Veja Também}
\begin{itemize}
	\item Introdução ao \LaTeX
		\\\url{http://latexbr.blogspot.com.br/2010/04/introducao-ao-latex.html}
	\item \LaTeX\ - A document preparation system
		\\\url{http://www.latex-project.org/}
	\item The \acrlong{CTAN}
		\\\url{http://ctan.org}
	\item \TeX Users Group
		\\\url{http://tug.org}
	\item \TeX\ - \LaTeX\ Stack Exchange
		\\\url{http://tex.stackexchange.com}
	\item \LaTeX\ Wikibook
		\\\url{http://en.wikibooks.org/wiki/LaTeX}
	\item write\LaTeX
		\\\url{http://www.writelatex.com}
\end{itemize}



%%%%%%%%%%%%%%%%%%%%%%%%%%%%%%%%%%%%%%%%%%%%%%%%%%%%%%%%%%%%%%%%%%%%%%%%%%%%%%%%
%%%%%%%%%%%%%%%%%%%%%%%%%%%%%%%%%%%%%%%%%%%%%%%%%%%%%%%%%%%%%%%%%%%%%%%%%%%%%%%%
%%%%%%%%%%%%%%%%%%%%%%%%%%%%%%%%%%%%%%%%%%%%%%%%%%%%%%%%%%%%%%%%%%%%%%%%%%%%%%%%
\section{Estrutura do Trabalho}%
O JusBrasil\footnote{\url{http://www.jusbrasil.com.br}} define plágio como
``reprodução, total ou parcial, da propriedade intelectual de alguém, inculcando-se
o criador da idéia ou da forma. Constitui crime contra a propriedade imaterial
violar direito de autor de obra literária, científica ou artística.''


A \acrfull{CEP} da Presidência da República decidiu  ``pela
aplicação de sanção ética aos servidores públicos que incorrerem na prática de
plágio"\footnote{\url{http://www.comissaodeetica.unb.br/index.php?view=article&id=8:plagio-academico}},
e a \acrfull{CAPES} recomenda que se adote políticas de conscientização e informação
sobre a propriedade intelectual, baseando-se na Proposição 2010.19.07379-01,
referente ao plágio nas instituições de ensino\footnote{\url{https://www.capes.gov.br/images/stories/download/diversos/OrientacoesCapes_CombateAoPlagio.pdf}}.


\subsection{Veja Também}
\begin{itemize}
	\item IEEE Plagiarism FAQ
		\\\url{http://www.ieee.org/publications_standards/publications/rights/plagiarism_FAQ.html}
	\item Relatório da Comissão de Integridade de Pesquisa do CNPq
		\\\url{http://www.cnpq.br/web/guest/documentos-do-cic}
\end{itemize}



%\section{Normas CIC}
% \href{http://monografias.cic.unb.br/dspace/normasGerais.pdf}{Política de Publicação de Monografias e Dissertações no Repositório Digital do CIC}%
% \href{http://monografias.cic.unb.br/dspace/}{Repositório do Departamento de Ciência da Computação da UnB}

% \href{http://bdm.bce.unb.br/}{Biblioteca Digital de Monografias de Graduação e Especialização}